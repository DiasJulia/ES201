\documentclass[12pt,a4paper]{article}

% --- Paquets de base ---
\usepackage[utf8]{inputenc}
\usepackage[french]{babel}
\usepackage[T1]{fontenc}
\usepackage{graphicx}
\usepackage{amsmath}
\usepackage{geometry}
\geometry{margin=2.5cm}

% --- Paquets pour tableaux et codes ---
\usepackage{booktabs}
\usepackage{listings}
\usepackage{xcolor}

% Configuration pour les blocs de code (gem5 stats)
\lstset{
    basicstyle=\ttfamily\small,
    keywordstyle=\color{blue},
    stringstyle=\color{red},
    commentstyle=\color{green!50!black},
    breaklines=true,
    frame=single,
    captionpos=b
}

% --- Informations du rapport ---
\title{Rapport de Travaux Pratiques \\ Architecture des Microprocesseurs \\ TD/TP4 : Évaluation de performances}
\author{Votre Nom Ici}
\date{\today}

\begin{document}

\maketitle

\section{Introduction}
Ce rapport présente les résultats obtenus lors du TP4, portant sur l'analyse de l'impact des configurations de mémoire cache et des fenêtres d'instruction sur les performances des processeurs.

\section{Exercice 3 : Évaluation de configurations de mémoires caches}

\subsection{Configurations de simulation}
Nous avons comparé deux configurations (C1 et C2) pour analyser l'impact de l'associativité sur le taux de défaut (miss rate).

\begin{table}[h!]
\centering
\caption{Paramètres des configurations C1 et C2.}
\begin{tabular}{@{}lll@{}}
\toprule
Paramètre & Configuration C1 & Configuration C2 \\ \midrule
L1-i (Taille/Assoc.) & 4KB / Direct Mapped & 4KB / Direct Mapped \\
L1-d (Tamanho/Assoc.) & 4KB / Direct Mapped & 4KB / 2-way \\
L2 (Tamanho/Assoc.)   & 32KB / Direct Mapped & 32KB / 4-way \\
Taille de bloc      & 32 Bytes & 32 Bytes \\ \bottomrule
\end{tabular}
\end{table}

\subsection{Résultats et Analyse}
L'analyse porte sur les différentes variantes de l'algorithme de multiplication de matrices (\textit{normal, pointer, tempo, unrol}). 

\begin{itemize}
    \item \textbf{Localité :} L'augmentation de l'associativité en C2 permet de réduire les conflits dans la cache de données.
    \item \textbf{Miss Rate :} [Insérez ici vos observations sur les résultats du simulateur].
\end{itemize}

\section{Exercice 4 : Analyse de performance (Cortex A7 vs A15)}

\subsection{Variation de la taille de cache L1}
Nous avons fait varier la taille de la cache L1 pour observer l'évolution de l'IPC (Instructions Per Cycle).

\subsection{Efficacité Surfacique et Énergétique}
En utilisant l'outil CACTI, nous avons calculé l'efficacité en fonction de la surface occupée.

\begin{equation}
\text{Efficacité Surfacique} = \frac{\text{IPC}}{\text{Surface (mm}^2)}
\end{equation}

\section{Conclusion}
Les simulations démontrent que le choix de la configuration de cache est un compromis entre performance (IPC) et coût matériel (surface/consommation). Pour le système \textit{big.LITTLE}, la configuration optimale serait...

\end{document}