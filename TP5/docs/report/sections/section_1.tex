\section*{Analyse théorique de cohérence de cache}

\subsection*{Q1 : En considérant que chaque thread s’exécute sur un processeur dans une architecture
de type multicoeurs à base de bus et 1 niveau de cache (comme décrit Figure 21), décrivez le
comportement de la hiérarchie mémoire et de la cohérence des caches pour l’algorithme de
multiplication de matrices. On supposera que le thread principal se trouve sur le processeur
d’indice 1.}

\section*{Paramètres de l'architecture multicoeurs}

\subsection*{Q2 : Examinez le fichier de déclaration d’un élément de type « processeur superscalaire out-
of-order », et présentez sous forme de tableau cinq paramètres configurables de ce type de
processeur avec leur valeur par défaut. Choisissez de préférence des paramètres étudiés lors
des séances TD/TP précédentes. Le fichier à consulter est le suivant:}

\begin{verbatim}
    $GEM5/src/cpu/o3/O3CPU.py
\end{verbatim}

\subsection*{Q3 : Examinez le fichier d’options de la plateforme se.py, puis déterminez et présentez sous
forme de tableau les valeurs par défaut des paramètres suivants :}
\begin{itemize}
    \item Cache de données de niveau 1 : associativité, taille du cache, taille de la ligne
    \item Cache d’instructions de niveau 1 : associativité, taille du cache, taille de la ligne
    \item Cache unifié de niveau 2 : associativité, taille du cache, taille de la ligne
\end{itemize}

Le fichier d’options à consulter est le suivant :

\begin{verbatim}
    $GEM5/configs/common/Options.py
\end{verbatim}