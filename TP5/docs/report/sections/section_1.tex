\section*{Analyse théorique de cohérence de cache}

\subsection*{Q1 : En considérant que chaque thread s’exécute sur un processeur dans une architecture
de type multicoeurs à base de bus et 1 niveau de cache (comme décrit Figure 21), décrivez le
comportement de la hiérarchie mémoire et de la cohérence des caches pour l’algorithme de
multiplication de matrices. On supposera que le thread principal se trouve sur le processeur
d’indice 1.}

\begin{itemize}
    \item \textbf{Matrice A :} Chaque ligne est stockée dans le cache d’un processeur différent chargé depuis la mémoire principale.
    \item Ainsi, chaque processeur a une ligne différente donc ne communique pas avec les autres, on a donc peu d'incohérence.
    \item Plus on a de thread moins on a d'incohérence.
    \item \textbf{Matrice B :} Elle est lue et stockée dans chaque thread pour calculer $C$ mais elle n’est pas modifiée donc nous n’avons pas non plus d'incohérence.
    \item \textbf{Matrice C :} Par définition du produit de matrice chaque ligne de $C$ correspond au produit de la ligne de $A$ correspondante avec $B$, ainsi, chaque processus écrit des lignes différentes de $C$ donc on n’a pas de problème de cohérence.
\end{itemize}

\section*{Paramètres de l'architecture multicoeurs}

\subsection*{Q2 : Examinez le fichier de déclaration d’un élément de type « processeur superscalaire out-
of-order », et présentez sous forme de tableau cinq paramètres configurables de ce type de
processeur avec leur valeur par défaut. Choisissez de préférence des paramètres étudiés lors
des séances TD/TP précédentes. Le fichier à consulter est le suivant:}

\begin{verbatim}
    $GEM5/src/cpu/o3/O3CPU.py
\end{verbatim}

\begin{table}[h]
\centering
\caption{Paramètres de configuration du processeur}
\begin{tabular}{|c|c|c|}
\hline
\textbf{Paramètre} & \textbf{Signification} & \textbf{Valeur par défaut} \\
\hline
fetchWidth & Nombre d’instructions récupérées par cycle & 8 \\
\hline
fetchBufferSize & Taille du buffer & 64 \\
\hline
decodeWidth & Nombre d'instructions décodées par cycle & 8 \\
\hline
cacheStorePorts & Nombre de ports disponibles pour accéder au cache (écriture) & 200 \\
\hline
issueWidth & Nombre d'instructions émises par cycle & 8 \\
\hline
\end{tabular}
\end{table}

\subsection*{Q3 : Examinez le fichier d’options de la plateforme se.py, puis déterminez et présentez sous
forme de tableau les valeurs par défaut des paramètres suivants :}
\begin{itemize}
    \item Cache de données de niveau 1 : associativité, taille du cache, taille de la ligne
    \item Cache d’instructions de niveau 1 : associativité, taille du cache, taille de la ligne
    \item Cache unifié de niveau 2 : associativité, taille du cache, taille de la ligne
\end{itemize}

Le fichier d’options à consulter est le suivant :

\begin{verbatim}
    $GEM5/configs/common/Options.py
\end{verbatim}

\begin{table}[h]
\centering
\caption{Spécifications des caches}
\begin{tabular}{|c|c|c|c|}
\hline
 & \textbf{L1 (l1d)} & \textbf{L1 (l1i)} & \textbf{L2 (l2)} \\
\hline
Taille du cache & 64kB & 32kB & 2MB \\
\hline
Taille de la ligne & 64B & 64B & 64B \\
\hline
Associativité & 2 & 2 & 8 \\
\hline
\end{tabular}
\end{table}